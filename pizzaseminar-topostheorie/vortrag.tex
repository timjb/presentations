\documentclass{article}

\usepackage[utf8]{inputenc}
\usepackage[ngerman]{babel}
\usepackage[a4paper]{geometry}

\geometry{margin=2cm}
\setlength{\parindent}{0pt}

\usepackage{amsmath,amsthm,amssymb}
\usepackage{stmaryrd} % \mapsfrom
\usepackage{enumitem}

\usepackage{tikz}
%\usetikzlibrary{matrix,arrows,cd}
\usetikzlibrary{matrix}

\theoremstyle{definition}
\newtheorem*{defn}{Definition}
\newtheorem*{satz}{Satz}
\newtheorem*{satzdefn}{Satz/Definition}
\newtheorem*{lem}{Lemma}
\newtheorem*{lemdefn}{Lemma/Definition}
\newtheorem*{bsp}{Beispiel}
\newtheorem*{bspe}{Beispiele}
\newtheorem*{kor}{Korollar}
\newtheorem*{prop}{Proposition}

\theoremstyle{remark}
\newtheorem*{erinnerung}{Erinnerung}
\newtheorem*{bem}{Bemerkung}
\newtheorem*{interp}{Interpretation}

% Miscellanea
\newcommand{\blank}{\text{--}} % Platzhalter
\newcommand{\coloneqq}{:=} % Definitionsgleich
\newcommand{\defequiv}{\vcentcolon\equiv}
\newcommand{\coloniff}{\enspace:\!\iff} % :<=> (Definitionsdoppelpfeil)
\newcommand{\?}{\,{:}\,}
\renewcommand{\_}{\mathpunct{.}\,}
% Schöne Mengen { #1 | #2 }
% siehe http://tex.stackexchange.com/questions/13634/define-pretty-sets-in-latex-esp-how-to-do-the-condition-separator
\usepackage{mathtools}
\DeclarePairedDelimiterX\Set[2]{\lbrace}{\rbrace}%
 { #1 \,\delimsize|\, #2 }
\newcommand{\N}{\mathbb{N}} % Natürliche Zahlen
\newcommand{\Powerset}{\mathcal{P}} % Potenzmenge (powerset)
\renewcommand{\O}{\mathcal{O}} % Verband der offenen Teilmengen

% Schöne Fürall- und Existenzquantoren
\newcommand{\fa}[1]{\forall \, {#1} .\,}
\newcommand{\ex}[1]{\exists \, {#1} .\,}
\newcommand{\exu}[1]{\exists! \, {#1} .\,} % Eindeutige Existenz

% Konzepte
\DeclareMathOperator{\Ob}{Ob} % Objekte (einer Kategorie)
\DeclareMathOperator{\End}{End} % Endomorphismenmenge / -klasse
\DeclareMathOperator{\Hom}{Hom} % Homomorphisms
\DeclareMathOperator{\id}{id} % Identität
\DeclareMathOperator{\Nat}{Nat} % Natürliche Transformationen
\DeclareMathOperator{\codom}{codom} % Codomain
\newcommand{\op}{\mathrm{op}} % opposite category
\newcommand{\ladj}{\dashv} % Links-adjungiert (left-adjoint)

% Topostheorie
\DeclareMathOperator{\Sub}{Sub} % Partialordnung der Subobjekte
\DeclareMathOperator{\true}{true} % universeller Monomorphismus
\newcommand{\clos}[1]{\overline{{#1}}} % "Abschluss" (im Lawvere-Tierney-Sinne)
\newcommand{\sheafification}{\mathbf{a}} % sheafification, Garbifizierung

% Populäre Kategorien
\newcommand{\SetC}{\mathbf{Set}} % Kategorie der Mengen
\newcommand{\FinSetC}{\mathbf{FinSet}} % Kategorie der endlichen Mengen
%\newcommand{\kAlg}{k\text{-}\Alg} % Kategorie der k-Algebren
\newcommand{\Sh}{\mathbf{Sh}} % Kategorie der Garben
\newcommand{\FuncC}[2]{[{#1}, {#2}]} % Funktorkategorie
\newcommand{\Etale}{\mathbf{\acute{E}t}} % étale Überlagerungen
\newcommand{\Bund}{\mathbf{Bund}} % Bündel

% Bezeichnungen für Variablen, die für Kategorien stehen
\newcommand{\Bat}{\mathcal{B}} % Category-B
\newcommand{\Cat}{\mathcal{C}} % Category-C
\newcommand{\Dat}{\mathcal{D}} % Category-D
\newcommand{\Eat}{\mathcal{E}} % Category-E
\newcommand{\Fat}{\mathcal{F}} % Category-F

% Bezeichnungen für Variablen, die für Garben stehen
\newcommand{\Fais}{\mathcal{F}} % Faisceau-F (Garbe auf französisch)

% Makros für Adjunktionen. Geklaut von
% http://sma.epfl.ch/~werndli/latex/adjunction.pdf
% http://sma.epfl.ch/~werndli/latex/adjunction.tex
\usepackage[all]{xy}
\newcommand{\adj}[1][]{\def\ArgI{#1}\adjRelayI}
\newcommand{\adjRelayI}[1][]{\def\ArgII{#1}\adjRelayII}
\newcommand{\adjRelayII}[3][2.2em]{
  \ensuremath{\SelectTips{lu}{10}\xymatrix@C=#1@1{
    {#2\,}
    \ar@<1ex>[r]^-{\ArgI}^-{}="1" &
    {\,#3}
    \ar@<1ex>[l]^-{\ArgII}^-{}="2"
    \ar@{}"1";"2"|(.3){\hbox{}}="7"
    \ar@{}"1";"2"|(.7){\hbox{}}="8"
    \ar@{|-} "8" ;"7"
  }}
}
\newcommand{\radj}[1][]{\def\ArgI{#1}\radjRelayI}
\newcommand{\radjRelayI}[1][]{\def\ArgII{#1}\radjRelayII}
\newcommand{\radjRelayII}[3][2.2em]{
  \ensuremath{\SelectTips{lu}{10}\xymatrix@C=#1@1{
  {#2\,}
  \ar@<-1ex>[r]_-{\ArgI}^-{}="1" &
  {\,#3}
  \ar@<-1ex>[l]_-{\ArgII}^-{}="2"
  \ar@{}"1";"2"|(.3){\hbox{}}="7"
  \ar@{}"1";"2"|(.7){\hbox{}}="8"
  \ar@{|-} "7" ;"8"
  }}
}

% http://tex.stackexchange.com/a/49196
\newcommand*\mycirc[1]{%
  \begin{tikzpicture}[baseline=(C.base)]
  \node[draw,circle,inner sep=1pt,minimum size=3ex](C) {#1};
  \end{tikzpicture}
}

\title{Geometrische Morphismen; Eigenschaften von Topoi und Konstruktionen mit Topoi}
\author{Tim Baumann}
\date{13. April 2017}

\begin{document}

\maketitle

\section{Geometrische Morphismen}

\begin{defn}
  Ein \emph{geometrischer Morphismus} $f : \Eat \to \Dat$ zwischen Topoi ist ein Paar
  \[ \radj[f_*][f^*]{\Eat}{\Dat} \]
  von adjungierten Funktoren, wobei $f^*$ linksexakt ist, d.\,h. $f^*$ bewahrt endliche Limiten. \\
  Dabei heißt $f^*$ \emph{Urbildfunktor} und $f_*$ \emph{Direktes-Bild-Funktor}.
\end{defn}

\begin{erinnerung}
  Außerdem bewahrt $f_*$ Limiten und $f^*$ Kolimiten, denn:
  \begin{center}
    \emph{
      Left-Adjoints Preserve Colimits (LAPC), \quad
      Right-Adjoints Preserve Limits (RAPL)
    }
  \end{center}
\end{erinnerung}

\begin{lem}
  Linksexakte Funktoren bewahren Monomorphismen.
\end{lem}

\begin{proof}
  Dies folgt aus der Äquivalenz: $f : X \to Y$ ist genau dann ein Monomorphismus, falls folgendes Diagramm ein Pullback-Diagram ist:

  \[
    \begin{tikzpicture}
      \matrix (mat) [matrix of nodes, column sep=2cm, row sep=0.8cm]{
        \node (X) {$X$}; &
        \node (X') {$X$}; \\
        \node (X'') {$X$}; &
        \node (Y) {$X$}; \\
      };
      \draw[->] (X) to node [above] {$\id_X$} (X');
      \draw[->] (X) to node [left] {$\id_X$} (X'');
      \draw[->] (X') to node [right] {$f$} (Y);
      \draw[->] (X'') to node [above] {$f$} (Y);
    \end{tikzpicture}
    \qedhere
  \]
\end{proof}

\begin{bem}
  Aus dem Yoneda-Lemma folgt: Bei einer Adjunktion $F \ladj G$ ist $F$ eindeutig (bis auf Isomorphie) durch~$G$ bestimmt (und umgekehrt).
  Die Daten eines geometrischen Morphismus sind also schon allein durch~$f^*$ oder~$f_*$ gegeben.
\end{bem}

\begin{bem}
  Geometrische Morphismen lassen sich komponieren,
  \[
    \radj[f_*][f^*]{\Eat}{\Dat} \! \radj[g_*][g^*]{}{\Fat}
    \quad \rightsquigarrow \quad
    \radj[g_* \circ f_*][f^* \circ g^*]{\Eat}{\Fat},
  \]
  da Paare adjungierter Funktoren komponieren und die Komposition $f^* \circ  g^*$ linksexakter Funktoren wieder linksexakt ist.
  So erhalten wir eine Kategorie der Topoi.
  Diese wird sogar zu einer 2-Kategorie, wenn wir definieren:
\end{bem}

\begin{defn}
  Ein \emph{Morphismus zwischen geometrischen Morphismen} $f, g : \Eat \to \Dat$ ist eine natürliche Transformation $\alpha : f^* \to g^*$.
\end{defn}

\begin{bem}
  Es gibt eine natürliche Bijektion $\Nat(f^*, g^*) \cong \Nat(g_*, f_*)$.
\end{bem}

\begin{bsp}[siehe {\cite[Abschnitt VII.1]{sigal}}]
  Sei $\Eat$ ein kovollständiger Topos (z.\,B. ein Grothendiecktopos).
  Dann hat der \emph{Globale-Schnitte-Funktor}
  \[
    \Gamma : \Eat \to \SetC, \quad
    E \mapsto \Hom_\Eat(1, E)
  \]
  einen Linksadjungierten, nämlich
  \[
    \Delta : \SetC \to \Eat, \quad
    S \mapsto \coprod_{s \in S} 1.
  \]
  Für diesen gelten
  \begin{align*} 
    \Delta(\{ \heartsuit \}) & =
    \coprod_{s \in \{ \heartsuit \}} 1 \cong 1 \qquad \text{und} \\
    \Delta(S \times T) & =
    \enspace \coprod_{\mathclap{(s, t) \in S \times T}} 1 \enspace \cong
    \left( \coprod_{s \in S} 1 \right) \times \left( \coprod_{t \in T} 1 \right) =
    \Delta(S) \times \Delta(T).
  \end{align*} 
  Außerdem kann man zeigen, dass~$\Delta$ auch Differenzkerne und somit alle endlichen Limiten erhält.
  Folglich ist
  \[ \radj[\Gamma][\Delta]{\Eat}{\SetC} \]
  ein geometrischer Morphismus.

  Es gibt auch keinen anderen geometrischen Morphismus $f : \Eat \to \SetC$, denn für jeden solchen Morphismus und jedes Objekt $E \in \Ob(\Eat)$ gilt:
  \[
    f_* E \cong
    \Hom_\SetC(1, f_* E) \cong
    \Hom_\Eat(f^* 1, E) \cong
    \Hom_\Eat(1, E) \cong
    \Gamma(E).
  \]
\end{bsp}

\begin{defn}
  Ein geometrischer Morphismus $f = (f^* \ladj f_*)$ heißt
  \begin{itemize}
    \item \emph{(geom.) Einbettung}, falls $f_*$ volltreu ist und
    \item \emph{(geom.) Surjektion}, falls $f^*$ volltreu ist.
  \end{itemize}
\end{defn}

\begin{bem}
  Jeder geometrische Morphismus lässt sich (bis auf Kategorienäquivalenz) eindeutig zerlegen in eine Surjektion gefolgt von einer Einbettung.
\end{bem}

\begin{defn}
  Ein \emph{Untertopos von~$\Eat$} ist ein Topos~$\Dat$ zusammen mit einer geometrischen Einbettung $\Dat \to \Eat$.
\end{defn}

\begin{bsp}
  $\FinSetC$ ist kein Untertopos von~$\SetC$ vermöge der Inklusion $i : \FinSetC \to \SetC$, denn es gibt keine endliche Menge $X$ mit
  \[ \Hom_\FinSetC(X, \{ \heartsuit, \diamondsuit \}) \cong \Hom_\SetC(\N, i(\{ \heartsuit, \diamondsuit \})), \]
  da die rechte Hom-Menge unendlich und die linke Hom-Menge für alle $X \in \Ob(\FinSetC)$ endlich ist.
  Somit besitzt~$i$ keinen Linksadjungierten.
\end{bsp}

\section{Stetige Abbildungen induzieren geometrische Morphismen}

Sei $f : X \to Y$ eine stetige Abbildung zwischen topologischen Räumen.

Dann erhalten wir den \emph{Direktes-Bild-Funktor}
\[
  f_* : \Sh(X) \to \Sh(Y), \quad
  \Fais \mapsto f_*(\Fais)
\]
mit $f_*(\Fais)(U) = \Fais(f^{-1}(U))$ für alle $U \subseteq Y$.

% II.9.1
\begin{lem}
  Sei $g : E \to Y$ ein étalér Raum.
  Dann ist $f^*(g) : f^*(E) \to X$ wieder étale.
\end{lem}

\begin{defn}
  Der \emph{Inverse-Bild-Funktor} ist die Komposition
  \[
    f^* \coloneqq \Sh(Y) \xrightarrow{\Lambda} \Etale(Y) \xrightarrow{f^*} \Etale(X) \xrightarrow{\Gamma} \Sh(X).
  \]
\end{defn}

\begin{lem}[{\cite[II.9.2]{sigal}}]
  $f^* \ladj f_*$  
\end{lem}

\begin{lem}[{\cite[II.9.3]{sigal}}]
  $f^* : \Sh(Y) \to \Sh(X)$ ist linksexakt
\end{lem}

\begin{proof}[Beweisskizze]
  Zurückziehen von étalén Räumen $f^* : \Etale(Y) \to \Etale(X)$ ist die Einschränkung vom Zurückziehen allgemeiner Bündel, $f^* : \Bund(Y) \to \Bund(X)$.
  Letzterer Funktor bewahrt kleine Limiten, da er rechtsadjungiert zu
  \[
    f_* : \Bund(X) \to \Bund(Y), \quad
    (E \xrightarrow{p} X) \mapsto (E \xrightarrow{f \circ p} Y)
  \]
  ist.
  Man zeigt nun, dass für jeden topologischen Raum~$Z$ die Unterkategorie $\Etale(Z)$ in $\Bund(Z)$ abgeschlossen unter der Bildung von endlichen Limiten ist.
\end{proof}

\begin{kor}
  $\radj[f_*][f^*]{\Sh(X)}{\Sh(Y)}$
  ist ein geometrischer Morphismus.
\end{kor}

% siehe Kapitel VII.1
\begin{prop}[siehe {\cite[Abschnitte VII.1 und IX.2]{sigal}}]
  Sei~$Y$ ein nüchterner topologischer Raum (z.\,B. ein Hausdorff-Raum in klassischer Metalogik).
  Dann gibt es eine 1-zu-1-Korrespondenz
  \[
    \begin{array}{r c l}
      \{ \text{ stetige Abbildungen $X \to Y$ } \} & \leftrightarrow & \{ \text{ geometrische Morphismen $\Sh(X) \to \Sh(Y)$ } \} \\
      f & \mapsto & (f^* \ladj f_*)
    \end{array}
  \]
\end{prop}

\begin{proof}[Beweisskizze]
  Da $f^*$ linksexakt ist, bildet $f^*$ Unterobjekte des terminalen Objektes auf Unterobjekte des terminalen Objekts ab.
  Da außerdem für jeden topol. Raum~$Z$ ein Isomorphismus $\Sub_{\Sh(Z)}(1) \cong \O(Z)$ (wobei $\O(Z)$ der Verband der offenen Teilmengen ist) existiert, erhalten wir eine Abbildung
  \[ f^* : \O(Y) \to \O(X). \]
  Wir definieren nun eine Abbildung $f : X \to Y$ durch
  \[
    f(x) \coloneqq y
    \,\, :\!\!\iff
    x \in f^*(V) \enspace \text{für alle offenen Umgebungen~$V$ von~$y$}.
    \qedhere
  \]
\end{proof}

\begin{defn}
  Ein \emph{Punkt} eines Topos~$\Eat$ ist ein geometrischer Morphismus $\SetC \to \Eat$.
\end{defn}

\begin{bem}
  Diese Definition ergibt Sinn, da für nüchterne topologische Räume~$X$ geometrische Morphismen $\SetC = \Sh(\{ \heartsuit \}) \to \Sh(X)$ in 1-zu-1-Korrespondenz zu stetigen Abbildungen $\{ \heartsuit \} \to X$, also zu Punkten von~$X$, stehen.
\end{bem}

% siehe IV.7. The slice Category as a Topos
\section{Scheibenkategorien von Topoi sind Topoi}

\begin{satz}["`Fundamentalsatz der Topostheorie"', {\cite[IV.7.1]{sigal}}] \mbox{}\\
  Sei $\Eat$ ein Topos, $B \in \Ob(\Eat)$.
  Dann ist auch die Scheibenkategorie $\Eat / B$ ein Topos.
\end{satz}

\begin{bem}
  Im Spezialfall $\Eat = \SetC$ lässt sich der Satz wie folgt beweisen:
  Sei $\Bat$ die diskrete Kategorie mit $\Ob(\Bat) = B$.
  Dann gibt es eine Kategorienäquivalenz $\SetC/B \simeq \FuncC{\Bat}{\SetC} \simeq \FuncC{\Bat^\op}{\SetC}$ gegeben durch
  \[
    \SetC/B \to \FuncC{\Bat}{\SetC}, \quad  
    (X \xrightarrow{p} B) \mapsto (b \in \Ob(\Bat) \mapsto p^{-1}(b)).
  \]
\end{bem}

\begin{proof}[Beweisskizze]
Wir müssen die Toposaxiome nachrechnen:
\begin{enumerate}[itemsep=10pt,label=\protect\mycirc{\arabic*}]
\item $\Eat/B$ ist endlich vollständig:

\begin{minipage}[t]{0.35 \linewidth}
Der Differenzkern in~$\Eat/B$ \\
berechnet sich wie in~$\Eat$:

\begin{tikzpicture}
  \matrix (mat) [matrix of nodes, column sep=1cm, row sep=1cm]{
    \node (K) {$K$}; &
    \node (X) {$X$}; &&
    \node (Y) {$Y$}; \\
    && \node (B) {$B$}; \\
  };
  \draw[->] (K) to node [above] {} (X);
  \draw[->, transform canvas={yshift=0.5ex}] (X) to node [above] {$f$} (Y);
  \draw[->, transform canvas={yshift=-0.5ex}] (X) to node [below] {$g$} (Y);
  \draw[->] (K) to node {} (B);
  \draw[->] (X) to node {} (B);
  \draw[->] (Y) to node {} (B);
\end{tikzpicture}
\end{minipage}
\begin{minipage}[t]{0.3 \linewidth}
Binäre Produkte in~$\Eat/B$ \\
sind Pullbacks in~$\Eat$:

\begin{tikzpicture}
  \matrix (mat) [matrix of nodes, column sep=1cm, row sep=0.6cm]{
    & \node (P) {$X \times_B Y$}; \\
    \node (X) {$X$}; &&
    \node (Y) {$Y$}; \\
    & \node (B) {$B$}; \\
  };
  \draw[->] (P) to node {} (Y);
  \draw[->] (P) to node {} (X);
  \draw[->] (P) to node {} (B);
  \draw[->] (X) to node {} (B);
  \draw[->] (Y) to node {} (B);
\end{tikzpicture}
\end{minipage}
\begin{minipage}[t]{0.35 \linewidth}
Das terminale Objekt in~$\Eat/B$ \\
ist der Morphismus $\id_B : B \to B$:

\begin{tikzpicture}
  \matrix (mat) [matrix of nodes, column sep=1.2cm, row sep=1cm]{
    \node (X) {$X$}; &&
    \node (B') {$B$}; \\
    & \node (B) {$B$}; \\
  };
  \draw[->] (X) to node [above] {$f$} (B');
  \draw[->] (X) to node [left] {$f$} (B);
  \draw[->] (B') to node [right] {$\id_B$} (B);
\end{tikzpicture}
\end{minipage}

\item $\Eat/B$ besitzt einen Unterobjektklassifizierer:

Sei $U : \Eat/B \to \Eat$ der offensichtliche Vergissfunktor.
Dieser Funktor ist linksadjungiert zum Funktor $(\blank \times B) : \Eat \to \Eat/B$, denn es gilt
\[ \Hom_\Eat(X, Y) \cong \Hom_{\Eat/B}(X \xrightarrow{f} B, Y \times B \xrightarrow{\pi_2} B) \]
natürlich in $(X \xrightarrow{f} B) \in \Ob(\Eat/B)$ und $Y \in \Ob(\Eat)$.
Da ein Morphismus~$f$ in~$\Eat/B$ genau dann ein Monomorphismus ist, wenn $U(f)$ ein solcher ist, gilt:
\[
  \Sub_{\Eat/B}(A \xrightarrow{f} B) \cong
  \Sub_\Eat(A) \cong
  \Hom_\Eat(A, \Omega_\Eat) \cong 
  \Hom_{\Eat/B}(A \xrightarrow{f} B, \Omega_\Eat \times B \xrightarrow{\pi_2} B)
\]
Nach Proposition I.3.1 in SiGaL ist folglich $\Omega_{\Eat/B} = \Omega_\Eat \times B \xrightarrow{\pi_2} B$ der Unterobjektklassifizierer in~$\Eat/B$.
Der universelle Monomorphismus $\true : 1 \to \Omega$ ist die Komposition

\begin{center}\begin{tikzpicture}
  \matrix (mat) [matrix of nodes, column sep=2cm, row sep=1cm]{
    \node (B) {$B$}; &
    \node (B') {$1 \times B$}; &
    \node (Omega) {$\Omega \times B$}; \\
    & \node (B'') {$B$}; \\
  };
  \draw[->] (B) to node [above] {${\cong}$} (B');
  \draw[->] (B') to node [above] {$\true_\Eat \times \id_B$} (Omega);
  \draw[->] (B) to node [left] {$\id_B$} (B'');
  \draw[->] (B') to node [right] {$\pi_2$} (B'');
  \draw[->] (Omega) to node [right] {$\pi_2$} (B'');
\end{tikzpicture}\end{center}

\item $\Eat/B$ ist kartesisch abgeschlossen:

Seien $(X \xrightarrow{p_X} B), (Y \xrightarrow{p_Y} B), (Z \xrightarrow{p_Z} B) \in \Ob(\Eat/B)$ gegeben.
Wir müssen ein Objekt $(W \xrightarrow{p_W} B)$ mit
\[ \Hom_{\Eat/B}(X \times_B Y, Z) \cong \Hom_{\Eat/B}(X, W) \]
finden.
In~$\SetC$ können wir dieses Objekt wie folgt konstruieren:
\[
  W = \coprod_{b \in B} Z_b^{Y_b}
  \quad \text{wobei} \quad
  Z_b \coloneqq p_Z^{-1}(b), \,
  Y_b \coloneqq p_Y^{-1}(b). \,
\]
Diese Definition können wir mit der internen Sprache von~$\Eat$ imitieren:
\[
  W =
  \left\{
    (b, G) \in B \times \Powerset(Y \times Z)
    \enspace\middle|
    \arraycolsep=2pt
    \begin{array}{rl}
      & G \subseteq (p_Y \circ \pi_Y)^{-1}(b) \cap (p_Z \circ \pi_Z)^{-1}(b) \\
      \wedge & \fa{y : Y} p_Y(y) = b \implies \exu{z : Z} (y, z) \in G
    \end{array}
  \right\}
\]
(Interpretation: Elemente von $W$ sind Tupel $(b, G)$ bestehend aus einem Index~$b \in B$ und dem Graphen eines Morphismus $Y_b \to Z_b$ als Teilmenge von $Y \times Z$.)
\qedhere

\end{enumerate}

\end{proof}

\begin{defn}
  Sei $k : B \to A$ ein Morphismus in~$\Eat$.
  Dann erhalten wir einen Funktor
  \[
    \Sigma_k : \Eat/B \to \Eat/A, \quad
    (X \xrightarrow{p_X} B) \mapsto (k \circ p_X : X \xrightarrow{p_X} B \xrightarrow{k} A)
  \]
  durch Komponieren mit~$k$ und einen \emph{Basiswechselfunktor}
  \[
    k^* : \Eat/A \to \Eat/B, \quad
    (Y \xrightarrow{p_Y} A) \mapsto (B \times_A Y \xrightarrow{\pi_B} B)
  \]
  durch Pullback entlang~$k$.
\end{defn}

\begin{bem}
  Im Spezialfall $\Eat = \SetC$, $k : B \to 1$ ist $k^*$ der Funktor, der eine Menge~$X$ auf die konstante Familie $(X)_{b \in B}$ von Mengen abbildet.
  Man kann $\Sigma_B \coloneqq \Pi_k : \FuncC{\Bat}{\SetC} \to \SetC$ wie folgt konstruieren:
  \[
    \Sigma_B((X_b)_{b \in B}) \coloneqq \{ \text{ Tupel $(b, x)$ mit $b \in B$ und $x \in X_b$ } \}
  \]
  In Typtheorie heißt diese Konstruktion "`abhängige Summe"' (oder auch "`$\Sigma$-Typ"').
\end{bem}

\begin{lem}[{\cite[I.9.4]{sigal}}]
  $\Sigma_k \ladj k^*$
\end{lem}

\begin{proof}
Wir müssen zeigen, dass
\[
  \Hom_{\Eat/A}(\Sigma_k(X \to B), Y \to A) \cong \Hom_{\Eat/B}(X \to B, k^*(Y \to A)).
\]
Betrachte das Diagramm
\begin{center}\begin{tikzpicture}
  \matrix (mat) [matrix of nodes, column sep=2cm, row sep=1cm]{
    \node (X) {$X$}; \\
    & \node (P) {$B \times_A Y$}; &
    \node (Y) {$Y$}; \\
    & \node (B) {$B$}; &
    \node (A) {$A$}; \\
  };
  \draw[->] (X) to node {} (B);
  \draw[->,dashed] (X) to node {} (P);
  \draw[->,dashed] (X) to node {} (Y);
  \draw[->] (P) to node {} (Y);
  \draw[->] (P) to node {} (B);
  \draw[->] (B) to node [above] {$k$} (A);
  \draw[->] (Y) to node {} (A);
\end{tikzpicture}\end{center}
Elemente der linken Hom-Menge sind Morphismen $X \to Y$, die das äußere Viereck kommutieren lassen; Elemente der rechten Hom-Menge sind Morphismen $X \to B \times_A Y$, die das linke Dreieck kommutieren lassen.
Zwischen solchen Elementen besteht eine 1-zu-1-Korrespondenz, gegeben durch die universelle Eigenschaft des Pullbacks~$B \times_A Y$.
\end{proof}

\begin{lemdefn}[{\cite[I.9.4]{sigal}}]
  $k^*$ besitzt auch einen Rechtsadjungierten $\Pi_k : \Eat/B \to \Eat/A$.
\end{lemdefn}

\begin{bem}
  Im Spezialfall $\Eat = \SetC$, $k : B \to 1$ kann $\Pi_B \coloneqq \Pi_k : \FuncC{\Bat}{\SetC} \to \SetC$ wie folgt konstruiert werden:
  \[
    \Pi_B((X_b)_{b \in B}) \coloneqq \{ \text{ Familien $(x_b \in X_b)_{b \in B}$ } \}
  \]
  In Typtheorie heißt diese Konstruktion "`abhängiges Produkt"' (oder auch "`$\Pi$-Typ"').
\end{bem}

\begin{proof}
  Wir dürfen annehmen, dass $A \cong 1$ und somit $\Eat/A \cong \Eat$. (Ansonsten verwende $\Eat' \coloneqq \Eat/A$ und $B' \coloneqq (B \xrightarrow{k} A) \in \Ob(\Eat')$ anstelle von~$\Eat$ bzw.~$B$. Beachte, dass $\Eat'/B' \simeq \Eat/B$.)
  Es gilt:
  \begin{align*}
    \Hom_{\Eat/B}(k^*(X), Y \xrightarrow{h} B)
    & \cong \Hom_{\Eat/B}(X \times B \xrightarrow{\pi_B} B, Y \xrightarrow{h} B) \\
    & \cong \Set{t \in \Hom_\Eat(X \times B, Y)}{h \circ t = \pi_B} \\
    & \cong \Set{t' \in \Hom_\Eat(X, Y^B)}{h^B \circ t' = j \circ {!}} \\
    & \cong \Hom_\Eat(X, \Set{g : Y^B}{h \circ g = \id_B})
  \end{align*}
  wobei $j : 1 \to B^B$ die Curryfizierung von $\id_B$ und $! : X \to 1$ ist.
  Wir definieren somit $\Pi_k$ durch
  \[
    \Pi_k(k : Y \to B) \coloneqq \Set{g : Y^B}{h \circ g = \id_B}.
    \qedhere
  \]
\end{proof}

\begin{kor}
  $\radj[\Pi_k][k^*]{\Eat/B}{\Eat/A}$
  ist ein geometrischer Morphismus. \\
  Dieser ist \emph{wesentlich}, d.\,h. $k^*$ besitzt auch einen Linksadjungierten.
\end{kor}

\begin{bem}
  Allquantifikation über $A$ lässt sich mit dem Scheibentopos $\Eat/A$ verstehen:
  \[
    \begin{array}{r l}
      & \Eat \models \fa{x:A} \varphi(x) \\
      \iff & 1 \models \fa{x:A} \varphi(x) \\
      \iff & \text{für alle $B \xrightarrow{p} 1$, $B \xrightarrow{x_0} A$ gilt} \enspace B \models \varphi(x_0) \\
      \iff & A \models \varphi(\id_A) \\
      \iff & \Eat/A \models (k^* \varphi)(\Delta), \\
    \end{array}
  \]
  wobei $k : A \to 1$ der eindeutige Morphismus und~$\Delta$ der Diagonalmorphismus in folgendem Diagramm ist:
  \begin{center}\begin{tikzpicture}
    \matrix (mat) [matrix of nodes, column sep=2cm, row sep=1cm]{
      \node (A) {$A$};
      && \node (A2) {$A \times A$}; \\
      & \node (A') {$A$}; \\
    };
    \draw[->] (A) to node [above] {$\Delta$} (A2);
    \draw[->] (A) to node [below] {$\id$} (A');
    \draw[->] (A2) to node [below] {$\pi_1$} (A');
  \end{tikzpicture}\end{center}
  Das verallgemeinerte Element $\id_A$ wird auch als \emph{generisches Element} von~$A$ bezeichnet, da man alle weiteren Elemente aus ihm durch Präkomposition erhalten kann.
\end{bem}

\section{Modale Operatoren}

\begin{defn}
  Ein \emph{modaler Operator} (oder \emph{Lawvere-Tierney-Topologie}) auf einem Topos~$\Eat$ ist ein Morphismus $\Box : \Omega \to \Omega$, für den gilt:
  \begin{center}
  \begin{minipage}[t]{0.2 \linewidth}
    (a) $\Box \circ \true = \true$

    \begin{tikzpicture}
      \matrix (mat) [matrix of nodes, column sep=1cm, row sep=1cm]{
        \node (One) {$1$}; &
        \node (Omega') {$\Omega$}; \\
        & \node (Omega'') {$\Omega$}; \\
      };
      \draw[->] (One) to node [above] {$\true$} (Omega');
      \draw[->] (One) to node [left] {$\true$} (Omega'');
      \draw[->] (Omega') to node [right] {$\Box$} (Omega'');
    \end{tikzpicture}
  \end{minipage}
  \begin{minipage}[t]{0.2 \linewidth}
    (b) $\Box \circ \Box = \Box$

    \begin{tikzpicture}
      \matrix (mat) [matrix of nodes, column sep=1cm, row sep=1cm]{
        \node (Omega) {$\Omega$}; &
        \node (Omega') {$\Omega$}; \\
        & \node (Omega'') {$\Omega$}; \\
      };
      \draw[->] (Omega) to node [above] {$\Box$} (Omega');
      \draw[->] (Omega) to node [left] {$\Box$} (Omega'');
      \draw[->] (Omega') to node [right] {$\Box$} (Omega'');
    \end{tikzpicture}
  \end{minipage}
  \begin{minipage}[t]{0.2 \linewidth}
    (b) $\Box \circ {\wedge} = \wedge \circ (\Box \times \Box)$

    \begin{tikzpicture}
      \matrix (mat) [matrix of nodes, column sep=1cm, row sep=1cm]{
        \node (Omega2) {$\Omega \times \Omega$}; &
        \node (Omega) {$\Omega$}; \\
        \node (Omega2') {$\Omega \times \Omega$}; &
        \node (Omega') {$\Omega$}; \\
      };
      \draw[->] (Omega2) to node [above] {$\wedge$} (Omega);
      \draw[->] (Omega2') to node [above] {$\wedge$} (Omega');
      \draw[->] (Omega2) to node [left] {$\Box \times \Box$} (Omega2');
      \draw[->] (Omega) to node [left] {$\Box$} (Omega');
    \end{tikzpicture}
  \end{minipage}
  \end{center}
\end{defn}

\begin{interp}
  $\Box$ ist ein idempotenter, mit ${\wedge}$ und $\true$ verträglicher modaler Operator.

  Zum Beispiel: Sei $\varphi$ eine Aussage. Wir könnten die logische Aussage $\Box \varphi$ wie folgt interpretieren:
  \begin{center}
    ich weiß, dass~$\varphi$ / ich glaube, dass~$\varphi$ / lokal gilt~$\varphi$
  \end{center}
  Für diese Interpretationen sollten intuitiv auch folgende Regeln gelten:
  \[
    \square \top = \top, \qquad
    \square \square \varphi \iff \square \varphi
    \qquad \text{sowie} \qquad
    (\square \varphi) \wedge (\square \psi) \iff \square (\varphi \wedge \psi)
  \]
  Diese obigen Interpretationen sollten sich also als modale Operatoren in passenden Topoi umsetzen lassen.
  Im Gegensatz dazu stiftet der modale Operator (im Sinne der Modallogik) $\lozenge$ mit der Interpretation "`$\lozenge \varphi$ gilt, falls $\varphi$ möglich ist"' keinen modalen Operator (im Sinne der Topostheorie), denn aus $(\lozenge \varphi) \wedge (\lozenge \psi)$ folgt i.\,A. nicht $\lozenge (\varphi \wedge \psi)$.
\end{interp}

\begin{defn}
  Für ein Unterobjekt $A \hookrightarrow E$ ist $\clos{A} \hookrightarrow E$ dasjenige Unterobjekt mit
  \[ \chi_{\clos{A}} = \Box \circ \chi_A : E \to \Omega. \]
\end{defn}

\begin{lem}
  \begin{minipage}[t]{0.8 \linewidth}\begin{itemize}
    \item
      $A \subseteq \clos{A}$, \quad
      $\clos{\clos{A}} = \clos{A}$, \quad
      $\clos{A \cap B} = \clos{A} \cap \clos{B}$
    \item $f^{-1}(\clos{A}) = \clos{f^{-1}(A)}$ \enspace (Natürlichkeit) \qquad
  \end{itemize}\end{minipage}
\end{lem}

\begin{defn}
  Sei $\Box$ ein modaler Operator auf~$\Eat$.
  \begin{itemize}[itemsep=0pt]
    \item Ein Unterobjekt $A \hookrightarrow E$ heißt \emph{dicht}, falls $\clos{A} = E$.
    \item
      Eine \emph{$\Box$-Garbe} ist ein Objekt~$F \in \Ob(\Eat)$, für das gilt: \\
      Für alle dichten Unterobjekte $A \xhookrightarrow{m} E$ ist $\Hom(m, F) : \Hom(E, F) \to \Hom(A, F)$ ein Isomorphismus.
    \item $\Sh_\Box(\Eat)$ ist die volle Unterkategorie der $\Box$-Garben von~$\Eat$.
  \end{itemize}
\end{defn}

\begin{satz}[{\cite[V.2.5]{sigal}}]
  $\Sh_\Box(\Eat)$ ist ein Topos.
\end{satz}

\begin{proof}[Beweisskizze]
  Zeige:
  \begin{itemize}
    \item Die Unterkategorie $\Sh_\Box(\Eat)$ ist abgeschlossen unter der Bildung von Limiten und Exponentialobjekten.
    \item
      Sei $\Omega_\Box$ der Differenzkern

      \begin{tikzpicture}
        \matrix (mat) [matrix of nodes, column sep=1cm, row sep=1cm]{
          \node (Omegaj) {$\Omega_\Box$}; &
          \node (Omega) {$\Omega$}; &
          \node (Omega') {$\Omega$}; \\
        };
        \draw[->] (Omegaj) to node {} (Omega);
        \draw[->, transform canvas={yshift=0.5ex}] (Omega) to node [above] {$\Box$} (Omega');
        \draw[->, transform canvas={yshift=-0.5ex}] (Omega) to node [below] {$\id$} (Omega');
      \end{tikzpicture}

      Nenne ein Subobjekt $A \hookrightarrow F$ \emph{abgeschlossen}, falls $\clos{A} = A$.
      Für eine $\Box$-Garbe~$F$ zeige dann, dass
      \[
        \Sub_{\Sh_\Box(\Eat)}(F) =
        \Set{A \in \Sub_\Eat(F)}{\text{$A$ abgeschlossen}} \cong
        \Hom_\Eat(F, \Omega_\Box)
      \]
      und dass $\Omega_\Box$ eine $F$-Garbe ist.
      Somit ist~$\Omega_\Box$ der Unterobjektklassifizierer von~$\Sh_\Box(\Eat)$. \qedhere
  \end{itemize}
\end{proof}

Sei $i : \Sh_\Box(\Eat) \to \Eat$ der Einbettungsfunktor.

\begin{satzdefn}[{\cite[V.3.1]{sigal}}]
  \begin{minipage}[t]{0.99 \linewidth}
    \begin{itemize}
      \item $i$ hat einen Linksadjungierten, die \emph{$\Box$-Garbifizierung} $\sheafification : \Eat \to \Sh_\Box(\Eat)$.
      \item $\sheafification$ ist linksexakt.
    \end{itemize}
  \end{minipage}
\end{satzdefn}

\begin{kor}
  $\radj[i][\sheafification]{\Sh_\Box(\Eat)}{\Eat}$
  ist eine geometrische Einbettung. \\
\end{kor}

\begin{bem}
  Bis auf Kategorienäquivalenz ist jede geometrische Einbettung von dieser Form.
\end{bem}

\begin{satz}[{\cite[V.1.2 und V.4.1]{sigal}}]
  Sei $\Cat$ eine kleine Kategorie und $\Eat \coloneqq \FuncC{\Cat^\op}{\SetC}$.
  Dann gibt es eine 1-zu-1-Korrespondenz
  \[
    \begin{array}{r c l}
      \{ \text{ Grothendieck-Topologien auf~$\Cat$ } \} & \leftrightarrow & \{ \text{ Lawvere-Tierney-Topologien auf~$\Eat$ } \} \\
      J & \mapsto & \Box_J \coloneqq (\Box_C : S \mapsto \Set{g}{\codom(g) = C, \text{ $S$ überdeckt $g$}})_{C \in \Ob(\Cat)} \\
      J_\Box \coloneqq \Box^*(1 \xhookrightarrow{\true} \Omega) & \mapsfrom & \Box
    \end{array}
  \]
  (Erinnerung: $\Omega \in \Ob(\Eat)$ ist die Prägarbe mit $\Omega(C) \coloneqq \{ \text{ Siebe auf~$C$ } \}$.)
\end{satz}

\begin{satz}[{\cite[V.4.2]{sigal}}]
  Desweiteren gilt für eine Prägarbe $P \in \Ob(\Eat)$:
  \[
    \text{$P$ ist $\Box$-Garbe}
    \iff
    \text{$P$ ist Garbe bzgl. $J_\Box$, also $P \in \Ob(\Sh(\Eat, J_\Box))$}.
  \]
\end{satz}

\begin{kor}
  Die Garbifizierungen einer Prägarbe bzgl. dem modalen Operator~$\Box$ oder der zugehörigen Grothendieck-Topologie~$J_\Box$ sind isomorph.
\end{kor}

% Achtung: Der Begriff einer Grothendieck-Topologie wurde im Seminar noch nicht definiert.
% Lukas hat am Anfang des zweiten Teils seines Vortrags den Begriff einer *Basis einer* Grothendieck-Topologie definiert.
% Dieser verhält sich zu Grothendieck-Topologien wie "Basis eines topol. Raums" zu "Topologie (eines topol. Raums)".

% aus Ingos Diss kopiert
\begin{defn}
Die~\emph{$\Box$-Übersetzung} ist rekursiv wie folgt definiert:
\newcommand{\optBox}{\textcolor{lightgray}{\Box}}
\begin{align*}
  (f = g)^\Box &\defequiv \Box(f = g) \\
  (x \in A)^\Box &\defequiv \Box(x \in A) \\
  \top^\Box &\defequiv \Box\top \quad \text{($\Leftrightarrow \top$)} \\
  \bot^\Box &\defequiv \Box\bot \\
  (\varphi \wedge \psi)^\Box &\defequiv \optBox(\varphi^\Box \wedge \psi^\Box) &
  \textstyle (\bigwedge_i \varphi_i)^\Box &\defequiv \textstyle \optBox(\bigwedge_i \varphi_i^\Box) \\
  (\varphi \vee \psi)^\Box &\defequiv \Box(\varphi^\Box \vee \psi^\Box) &
  \textstyle (\bigvee_i \varphi_i)^\Box &\defequiv \textstyle \Box(\bigvee_i \varphi_i^\Box) \\
  (\varphi \Rightarrow \psi)^\Box &\defequiv \optBox(\varphi^\Box \Rightarrow \psi^\Box) \\
  (\forall x\?X\_ \varphi)^\Box &\defequiv \optBox(\forall x\?X\_ \varphi^\Box) &
  (\forall X\_ \varphi)^\Box &\defequiv \optBox(\forall X\_ \varphi^\Box) \\
  (\exists x\?X\_ \varphi)^\Box &\defequiv \Box(\exists x\?X\_ \varphi^\Box) &
  (\exists X\_ \varphi)^\Box &\defequiv \Box(\exists X\_ \varphi^\Box)
\end{align*}
\end{defn}

% TODO: Quelle hierfür?
\begin{satz}
  Sei $\varphi$ eine Formel in $\Eat$.
  Dann gilt
  \[
    \Eat \models \varphi^\Box \iff
    \Sh_\Box(\Eat) \models \varphi,
  \]
  wobei in~$\varphi$ auf der linken Seite alle Parameter nach $\Sh_\Box(\Eat)$ zurückgezogen wurden.
\end{satz}

\begin{bspe}
  Sei $\psi$ eine logische Aussage, also ein Subobjekt $\psi : J \to 1$.
  \begin{itemize}
    \item
      $\Box \varphi \defequiv (\psi \implies \varphi)$ \quad
      (hiermit wird $\Sh_\Box(\Eat)$ zu einem \emph{offenen Untertopos} von~$\Eat$)
    \item
      $\Box \varphi \defequiv (\psi \vee \varphi)$ \quad
      (hiermit wird $\Sh_\Box(\Eat)$ zu einem \emph{abgeschlossenen Untertopos} von~$\Eat$)
    \item
      $\Box \varphi \defequiv \neg \neg \varphi$ \quad
      ($\Sh_{\neg \neg}(\Eat)$ ist der größte Untertopos von~$\Eat$, in dem klassische Logik gilt.)
  \end{itemize}
\end{bspe}

Betrachte $\Eat = \Sh(X)$, den Topos der Garben auf einem topologischen Raum.
Dann entsprechen Subobjekte $J \hookrightarrow 1$ genau den offenen Teilmengen $U \subseteq X$.
\begin{itemize}
  \item
    Für $\Box \varphi \defequiv (U \implies \varphi)$ ist $\Sh_\Box(X) \equiv \Sh(U)$ und es gilt
    \[
      \Sh(X) \models \varphi^\Box \iff
      \Sh(U) \models \varphi.
    \]
  \item
    Für $A \subseteq X$ abgeschlossen sei $\Box \varphi \defequiv ((X \setminus A) \vee \varphi)$.
    Es gilt
    \[
      \Sh(X) \models \varphi^\Box \iff
      \Sh(A) \models \varphi.
    \]
\end{itemize}

\section{Geometrische Morphismen und Logik}

\begin{defn}
  \begin{itemize}
  \item
    Eine Formel heißt \emph{geometrisch}, falls sie nur folgende Operatoren enthält:
    \[
      \top \quad
      \bot \quad
      \wedge \quad
      \vee \quad
      \exists \quad
      \bigvee
    \]
    Es sind also $\Rightarrow$ und $\forall$ verboten. Freie Variablen dürfen in~$\varphi$ vorkommen.
  \item
    Eine \emph{geometrische Sequenz} ist eine Sequenz der Form $\varphi \vdash \psi$, wobei~$\varphi$ und~$\psi$ geometrische Formeln sind.
  \end{itemize}
\end{defn}

% TODO: Quelle?
\begin{satz}
  Sei $f : \Eat \to \Dat$ ein geometrischer Morphismus.
  \begin{itemize}
  \item Gilt $\varphi \vdash \psi$ in~$\Dat$, so gilt $f^*(\varphi) \vdash f^*(\psi)$ in~$\Eat$.
  \item Ist $f$ eine Surjektion, so gilt auch die Umkehrung.
  \end{itemize}
\end{satz}

\section{Weitere Quellen für geometrische Morphismen}

% VII.2.2
\begin{satz}[{\cite[VII.2.2]{sigal}}]
  Sei $\phi : \Cat \to \Dat$ ein Funktor.
  Dann gibt es einen geometrischen Morphismus \\
  $\radj[\phi_*][\phi^*]{\FuncC{\Cat^\op}{\SetC}}{\FuncC{\Dat^\op}{\SetC}}$
  mit $\phi^*(P) \coloneqq (\Cat^\op \xrightarrow{\phi^\op} \Dat^\op \xrightarrow{P} \SetC)$.
  Dieser ist wesentlich. % d.\,h. $\phi^*$ besitzt auch einen Linksadjungierten
\end{satz}

\begin{thebibliography}{9}

\bibitem[SiGaL]{sigal}
  Saunders MacLane, Ieke Moerdijk,
  \emph{Sheaves in Geometry and Logic},
  Springer-Verlag, New York,
  1st edition,
  1992.

\end{thebibliography}

\end{document}

TODO: warum sind Punkte interessant?
TODO: Grothendieck-Topologie zu neg-neg (?)
TODO: Untertopoi von Grothendiecktopoi sind Grothendiecktopoi (?)
TODO: Punkte von Grothendieck-Topoi (?)
TODO: Logische Morphismen (?)
TODO: Verband der Subtopoi (?)
TODO: logische Interpretation: "Ein geom. Mor. F \to E macht F zu einem E-Topos; aus interner Sicht von E wird diese Situation zu F -> Set"
TODO: Eigenschaften von geometrischen Morphismen: offen/abgeschlossen
