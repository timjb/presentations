\documentclass{beamer}

\usepackage[utf8]{inputenc}
\usepackage[ngerman]{babel}

\mode<presentation> {
  %\setbeameroption{show notes} % Kommentare im fertigen PDF

  \usetheme{Boadilla}
  \usecolortheme{seagull}

  \setbeamertemplate{footline}[page number] % Einfacher Folienzähler als Fußzeile
  \setbeamertemplate{navigation symbols}{} % Keine Navigationslinks
}

\title{Theoreme für lau!}
\author{Tim Baumann}
\institute[CCA]{Curry Club Augsburg}
\date{16. Mai 2017}

\usepackage{proof}

%\usepackage[outputdir=output]{minted} % Syntax-Highlighted Code; requires pygments to be installed
%\newminted[frankcode]{agda}{}
%\newmintinline[frankinline]{haskell}{}

\usepackage{hyperref}
%\usepackage{graphicx}

\newcommand{\defeq}{:=} % Definitionsgleich
\newcommand{\defiff}{:\Leftrightarrow} % Definitionsäquivalent
\newcommand{\IsType}[1]{{#1}\,\,\text{type}}
\newcommand{\IsCtx}[1]{{#1}\,\,\text{ctx}}
\newcommand{\Bool}{\text{Bool}}
\newcommand{\trueV}{\text{true}}
\newcommand{\falseV}{\text{false}}
\newcommand{\fa}[1]{\forall {#1}.\,}
\newcommand{\lam}[1]{\lambda #1.\,}
\newcommand{\Lam}[1]{\Lambda #1.\,}
\newcommand{\Types}{\text{Types}}
\newcommand{\Terms}{\text{Terms}}
\newcommand{\emptyCtx}{\bullet}
\newcommand{\blank}{\text{--}} % Platzhalter

\usepackage{mathtools}
\usepackage{stmaryrd}
\DeclarePairedDelimiterX\Set[2]{\lbrace}{\rbrace}{ #1 \,\delimsize|\, #2 }
\DeclarePairedDelimiterX\interp[1]{\llbracket}{\rrbracket}{ #1 }

\newcommand{\typeInterp}[2]{\interp{#2}_{#1}}
\newcommand{\termInterp}[3]{\interp{#3}_{#1, #2}}
\newcommand{\relInterp}[2]{\interp{#2}_{#1}}
\newcommand{\Rel}[3]{#1 : #2 \Leftrightarrow #3}

% Populäre Typen
\DeclareMathOperator{\Sum}{Sum}
\DeclareMathOperator{\Product}{Product}
\DeclareMathOperator{\List}{List}
\DeclareMathOperator{\Nat}{Nat}

% Färbe \emph{} violett
\usepackage{color,xcolor,graphicx,overpic}
\definecolor{Emph}{rgb}{0.2,0.2,0.8}
\renewcommand{\emph}[1]{\textcolor{Emph}{#1}}

\definecolor{dimgray}{rgb}{0.41, 0.41, 0.41}
\newcommand{\info}[1]{\textcolor{dimgray}{#1}}

\newtheorem*{satz}{Satz}

% Macros copied from
% https://github.com/jonsterling/latex-common-notation
% https://github.com/jonsterling/forcing-bar-induction-in-system-t/blob/2bc8c69d849f03579470e5668e6b2a98af6c8054/macros.sty
\newlength{\jmssavedboxsep}
\newcommand\DeclBox[1]{%
  \setlength{\jmssavedboxsep}{\fboxsep}%
  \setlength{\fboxsep}{1.5pt}%
  \fcolorbox{black!20}{white}{$\displaystyle #1$}%
  \setlength{\fboxsep}{\jmssavedboxsep}%
}

\newcommand\DeclJdg[2]{\DeclBox{#1 \textit{ vorausgesetzt } #2}}

\begin{document}

\begin{frame}
  \titlepage
\end{frame}

\begin{frame}
  \frametitle{System F a.k.a. Girard–Reynolds polymorphic $\lambda$-calculus}

  Ein \emph{Typkontext} ist eine endliche Menge $\Delta = \{ X_1, \ldots, X_n \}$ von Typvariablen.
  Die Typen sind durch die induktive Definition
  \[
    \tau \defeq X \,|\, \Bool \,|\, \tau_1 \to \tau_2 \,|\, \fa{X} \tau'
  \]
  gegeben.
  Die Typen im Typkontext~$\Delta$ sind die Typen, deren freie Variablen in~$\Delta$ liegen. Formal:
  \begin{gather*}
    \DeclBox{\Delta \vdash \IsType{\tau}} \qquad
    \infer{
      \Delta \vdash \IsType{\Bool}
    }{} \qquad
    \infer{
      \Delta \vdash \IsType{X}
    }{
      X \in \Delta
    } \\[0.4cm]
    \infer{
      \Delta \vdash \IsType{\tau_1 \to \tau_2}
    }{
      \Delta \vdash \IsType{\tau_1} \quad
      \Delta \vdash \IsType{\tau_2}
    } \qquad
    \infer{
      \Delta \vdash \IsType{\fa{X} \tau}
    }{
      \Delta \cup \{ X \} \vdash \IsType{\tau}
    }
  \end{gather*}
\end{frame}

\begin{frame}
  \frametitle{Church-Kodierung in System F}
  \[\begin{array}{r c l}
    \Product(A, B) &\defeq& \fa{Y} (A \to B \to Y) \to Y \\
    \Sum(A, B) &\defeq& \fa{Y} (A \to Y) \to (B \to Y) \to Y \\
    \Nat &\defeq& \fa{Y} Y \to (Y \to Y) \to Y \\
    \List(A) &\defeq& \fa{Y} Y \to (A \to Y \to Y) \to Y
  \end{array}\]
\end{frame}

\begin{frame}
  \frametitle{Terme in System F}
  Ein \emph{Wertekontext} ist eine endliche Menge $\Gamma = \{ x_1 : \tau_1, \ldots, x_m : \tau_m \}$, wobei $x_1, \ldots, x_m$ paarweise verschiedene \emph{Typvariablen} sind und $\tau_1, \ldots, \tau_m$ Typen sind.
  Er bildet zusammen mit~$\Delta$ einen \emph{Kontext} $\Delta; \Gamma$ falls die freien Variablen der Typen $\tau_1, \ldots, \tau_m$ in~$\Delta$ liegen. Formal:
  \[
    \DeclBox{\IsCtx{(\Delta; \Gamma)}} \qquad
    \infer{
      \IsCtx{(\Delta; {\emptyCtx})}
    }{} \qquad
    \infer{
      \IsCtx{(\Delta; \Gamma \cup \{x : \tau\})}
    }{
      \IsCtx{(\Delta; \Gamma)} \qquad
      \Delta \vdash \IsType{\tau}
    }
  \]
\end{frame}

\begin{frame}
  \frametitle{Terme in System F}
  Die Terme~$t$ in System F sind induktiv definiert durch
  \[
    t \defeq \trueV \,|\, \falseV \,|\, \text{\textbf{if }} t_1 \text{\textbf{ then }} t_2 \text{\textbf{ else }} t_3 \,|\, \lam{x{:}\tau} t \,|\, t_1\,t_2 \,|\, \Lam{A} t \,|\, t\,\tau
  \]
  Ein Term $t$ hat Typ~$\tau$ in einem Kontext $\Delta; \Gamma$ (notiert $\Delta; \Gamma \vdash t : \tau$), falls
  \[ \DeclJdg{\Delta; \Gamma \vdash t : \tau}{\IsCtx{(\Delta; \Gamma)}} \qquad \]
  \[
    \infer{
      \Delta; \Gamma \vdash x : \tau
    }{
      (x : \tau) \in \Gamma
    }
  \]
  \[\begin{array}{c c}
    \infer{
      \Delta; \Gamma \vdash v : \Bool
    }{
      v \in \{ \trueV, \falseV \}
    } &
    \infer{
      \Delta; \Gamma \vdash \text{\textbf{if }} b \text{\textbf{ then }} t \text{\textbf{ else }} e : \tau
    }{
      \Delta; \Gamma \vdash b : \Bool \quad
      \Delta; \Gamma \vdash t : \tau \quad
      \Delta; \Gamma \vdash e : \tau
    } \\[0.3cm]
    \infer{
      \Delta; \Gamma \vdash \lam{x{:}\tau_1} t : \tau_1 \to \tau_2
    }{
      \Delta; \Gamma \cup \{ x : \tau_1 \} \vdash t : \tau_2
    } &
    \infer{
      \Delta; \Gamma \vdash t_1\,t_2 : \tau_2
    }{
      \Delta; \Gamma \vdash t_1 : \tau_1 \to \tau_2 \quad
      \Delta; \Gamma \vdash t_2 : \tau_1
    } \\[0.3cm]
    \infer{
      \Delta; \Gamma \vdash \Lam{A} t : \fa{A} \tau
    }{
      \Delta \cup \{ A \}; \Gamma \vdash t : \tau
    } &
    \infer{
      \Delta; \Gamma \vdash t\,\tau : \tau'[\tau/A]
    }{
      \Delta; \Gamma \vdash t : \fa{A} \tau' \quad
      \Delta \vdash \IsType{\tau}
    }
  \end{array}\]
\end{frame}

\begin{frame}
  \frametitle{Beispielterme}
  \[\begin{array}{r l l}
    \text{null} &:& \fa{A} \List(A) \to \Bool \\
    \text{null} &\defeq& \Lam{A} \lam{xs{:}\List(A)} \\
    && \quad xs\,\Bool\,\trueV\,(\lam{a{:}A} \lam{b{:}\Bool} \falseV)
  \end{array}\]
  \[\begin{array}{r l l}
    \text{pair} &:& \fa{A} \fa{B} A \to B \to \Product(A, B) \\
    \text{pair} &\defeq& \Lam{A} \Lam{B} \lam{a{:}A} \lam{b{:}B} \\
    && \quad \Lam{Y} \lam{f{:}A \to B \to Y} \\
    && \quad\quad f\,a\,b
  \end{array}\]
  \[\begin{array}{r l l}
    \text{append} &:& \fa{A} \List(A) \to \List(A) \to \List(A) \\
    \text{append} &\defeq& \Lam{A} \lam{xs{:}\List(A)} \lam{ys{:}\List(A)} \\
    && \quad \Lam{Y} \lam{y{:}Y} \lam{f{:}A \to Y \to Y} \\
    && \quad\quad ys\,Y\,(xs\,Y\,y\,f)\,f
  \end{array}\]
\end{frame}

\begin{frame}
  \frametitle{Interpretation von Typen}

  Eine \emph{Typumgebung} für einen Typkontext~$\Delta$ ist eine Abbildung
  \[ \vec{A} : \Delta \to \Types_\emptyCtx \]
  (wobei $\Types_{\widetilde{\Delta}} \defeq \Set{\tau}{\widetilde{\Delta} \vdash \tau}$)

  Jede Typumg. $\vec{A}$ induziert für jeden disjunkten Typkontext~$\Delta'$ eine Abb.
  \[ \typeInterp{\vec{A}}{\blank} : \Types_{\Delta \sqcup \Delta'} \to \Types_{\Delta'} \]
  rekursiv definiert durch
  \[
    \begin{array}{r c l l}
      \typeInterp{\vec{A}}{\Bool} &\defeq& \Bool \\
      \typeInterp{\vec{A}}{X} &\defeq& \vec{A}(X) \text{ falls $X \in \Delta$} \\
      \typeInterp{\vec{A}}{X} &\defeq& X \text{ falls $X \in \Delta'$} \\
      \typeInterp{\vec{A}}{\tau_1 \to \tau_2} &\defeq& \typeInterp{\vec{A}}{\tau_1} \to \typeInterp{\vec{A}}{\tau_2} \\
      \typeInterp{\vec{A}}{\fa{X} \tau} &\defeq& \fa{X} \typeInterp{\vec{A}}{\tau} & \text{\info{(\OE{} $X \not\in \Delta \cup \Delta'$)}} \\
    \end{array}
  \]
\end{frame}

\begin{frame}
  \frametitle{Interpretation von Termen}

  Eine \emph{Termumgebung} $\vec{a}$ für einen Kontext $\Delta; \Gamma$ bzgl. einer Typumgebung $\vec{A}$ für~$\Delta$ ist eine Familie
  \[ (\vec{a}(x) : \typeInterp{\vec{A}}{\tau})_{(x : \tau) \in \Gamma}. \]

  von Termen.
  Jede solche Termumgebung induziert für jeden von $\Delta;\Gamma$ disjunkten Kontext $\Delta';\Gamma'$ eine Abbildung
  \[ \termInterp{\vec{A}}{\vec{a}}{\blank} : \Terms_{\Delta \sqcup \Delta', \Gamma \sqcup \Gamma'} \to \Terms_{\Delta', \Gamma'} \]
  rekursiv definiert durch
  \[
    \begin{array}{r c l l}
      \termInterp{\vec{A}}{\vec{a}}{v} &\defeq& v & \info{(v \in \{ \trueV, \falseV \})} \\
      \termInterp{\vec{A}}{\vec{a}}{\text{\textbf{if }} b \text{\textbf{ then }} t \text{\textbf{ else }} e} &\defeq& \text{\textbf{if }} \termInterp{\vec{A}}{\vec{a}}{b} \text{\textbf{ then }} \termInterp{\vec{A}}{\vec{a}}{t} \!\!&\!\! \text{\textbf{ else }} \termInterp{\vec{A}}{\vec{a}}{e} \\
      \termInterp{\vec{A}}{\vec{a}}{\lam{x{:}\tau} t} &\defeq& \lam{x{:}\typeInterp{\vec{A}}{\tau}} \termInterp{\vec{A}}{\vec{a}}{t} & \text{\info{(\OE{} $x \not\in \Gamma \cup \Gamma'$)}} \\
      \termInterp{\vec{A}}{\vec{a}}{t_1 \, t_2} &\defeq& \termInterp{\vec{A}}{\vec{a}}{t_1} \, \termInterp{\vec{A}}{\vec{a}}{t_2} \\
      \termInterp{\vec{A}}{\vec{a}}{\Lam{X} t} &\defeq& \Lam{X} \termInterp{\vec{A}}{\vec{a}}{t} & \text{\info{(\OE{} $X \not\in \Delta \cup \Delta'$)}} \\
      \termInterp{\vec{A}}{\vec{a}}{t\,\tau} &\defeq& \termInterp{\vec{A}}{\vec{a}}{t}\,\typeInterp{\vec{A}}{\tau} \\
    \end{array}
  \]
\end{frame}

\begin{frame}
  \frametitle{Relationen zwischen Typen}

  Eine \emph{Relation} $\Rel{\mathcal{A}}{A}{A'}$ zwischen zwei Typen $A, A' \in \Types_\bullet$ ist eine Relation zwischen den Termen dieser beiden Typen, für die gilt:
  \[\begin{array}{r c l}
    \mathcal{A}(t_1[s_1/x_1], t_2[s_2/x_2]) &\implies&
    \mathcal{A}((\lam{x_1:\tau_1} t_1)\,s_1, (\lam{x_2:\tau_2} t_2)\,s_2) \\
    \mathcal{A}(t_1[\tau_1/X_1], t_2[\tau_2/X_2]) &\implies&
    \mathcal{A}((\Lam{X_1} t_1)\,\tau_1, (\Lam{X_2} t_2)\,\tau_2)
  \end{array}\]
  für alle passenden $X_i, \tau_i, t_i, s_i$ sodass
  \[
    \emptyCtx \vdash (\lam{x_i:\tau_i} t_i)\,s_i : A
    \quad \text{bzw.} \quad
    \emptyCtx \vdash (\Lam{X_i} t_i)\,\tau_i : A
    \qquad (i = 1,2).
  \]
  TODO: verbessern!!!

  Eine \emph{Relation}~$\Rel{\vec{\mathcal{A}}}{\vec{A}}{\vec{A'}}$ zwischen Typumgebungen $\vec{A}$ und $\vec{A'}$ für~$\Delta$ ist eine Familie
  \[ (\Rel{\mathcal{A}_X}{\vec{A}(X)}{\vec{A'}(X)})_{X \in \Delta} \]
  von Relationen zwischen den Typen $\vec{A}(X)$ und $\vec{A'}(X)$
\end{frame}

\begin{frame}
  \frametitle{Relationen zwischen Typen}
  Solch eine Relation $\Rel{\vec{\mathcal{A}}}{\vec{A}}{\vec{A'}}$ induziert für jeden Typ~$\tau$ im Typkontext~$\Delta$ eine Relation $\Rel{\relInterp{\vec{\mathcal{A}}}{\tau}}{\typeInterp{\vec{A}}{\tau}}{\typeInterp{\vec{A'}}{\tau}}$ wie folgt:
  \[
    \begin{array}{r c l}
      \relInterp{\mathcal{A}}{\Bool} &\defeq& \Bool \\
      \relInterp{\mathcal{A}}{X} &\defeq& \mathcal{A}_X \\
      \relInterp{\mathcal{A}}{\tau_1 \to \tau_2} &\defeq& \relInterp{\mathcal{A}}{\tau_1} \to \relInterp{\mathcal{A}}{\tau_2} \\
      \relInterp{\mathcal{A}}{\fa{X} \tau} &\defeq& \sim \text{ mit } g \sim g' \text{ genau dann wenn} \\
      && \text{für alle $A, A' \in \Types_\emptyCtx$} \\
      && \text{und Relationen $\Rel{\mathcal{A}}{A}{A'}$} \\
      && \text{gilt } \relInterp{(\vec{\mathcal{A}} \cup \{X \mapsto \mathcal{A}\})}{\tau}(g\,A, g'\,A')
    \end{array}
  \]
  mit $(\mathcal{R}_1 \to \mathcal{R}_2)(f, f') \defiff \text{für alle $a$, $a'$ mit $\mathcal{R}_1(a, a')$ gilt $\mathcal{R}_2(f\,a, f'\,a')$}$.
\end{frame}

\begin{frame}
  \frametitle{Parametrizität}

  Seien $\vec{A}$ und $\vec{A'}$ Typumgebungen für~$\Delta$, $\Rel{\vec{\mathcal{A}}}{\vec{A}}{\vec{A'}}$ eine Relation.
  Zwei Termumg. $\vec{a}$ und $\vec{a'}$ für $\Delta; \Gamma$ bzgl. $\vec{A}$ bzw.~$\vec{A'}$ sind \emph{relatiert} bzgl. $\vec{\mathcal{A}}$, falls
  \[
    \relInterp{\vec{\mathcal{A}}}{\tau}(\vec{a}(x), \vec{a'}(x))
    \quad \text{für alle $(x : \tau) \in \Gamma$}.
  \]
  \[
    \arraycolsep=2pt
    \begin{array}{r c l l}
    \Delta; \Gamma \models t : \tau &\defiff
    &\text{für}& \text{alle Typumgebungen $\vec{A}$, $\vec{A'}$ von~$\Delta$ und} \\
    &&& \text{alle Relationen $\Rel{\vec{\mathcal{A}}}{\vec{A}}{\vec{A'}}$ und} \\
    &&& \text{alle relatierten Termumg. $\vec{a}$ bzgl.~$\vec{A}$ und $\vec{a'}$ bzgl. $\vec{A'}$} \\
    &&\text{gilt}& \relInterp{\vec{\mathcal{A}}}{\tau}(\termInterp{\vec{A}}{\vec{a}}{t}, \termInterp{\vec{A'}}{\vec{a'}}{t})
  \end{array}
  \]

  \begin{satz}[Parametrizität]
    Aus $\Delta; \Gamma \vdash t : \tau$ folgt $\Delta; \Gamma \models t : \tau$
  \end{satz}

  Kurz: Interpretationen in relatierten Umgebungen sind relatiert.
\end{frame}

\begin{frame}
  TODO:
  \begin{itemize}
    \item Beispiele für freie Theoreme
    \item Ist ein primitiver Typ (wie Bool) überhaupt nötig?
  \end{itemize}
\end{frame}

\end{document}