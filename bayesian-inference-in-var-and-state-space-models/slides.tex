\documentclass[10pt]{beamer}

\mode<presentation>
{
  \usetheme{default}
  \usecolortheme{default}
  \usefonttheme{default}
  \useoutertheme[]{smoothbars}
  \setbeamertemplate{navigation symbols}{}
  \setbeamertemplate{caption}[numbered]
} 

\usepackage[utf8x]{inputenc}
\usepackage[ngerman]{babel}

\usepackage{color}
\usepackage{mathtools}

\theoremstyle{definition}
\newtheorem*{bsp}{Beispiel}
\newtheorem*{prior}{Prior-Verteilung}
\newtheorem*{inference}{Inferenz}

\newcommand{\R}{\mathbb{R}} % Reelle Zahlen
\DeclareMathOperator{\var}{Var} % Varianz
%\DeclareMathOperator{\cov}{Cov} % Kovarianz
%\DeclareMathOperator{\cor}{Cor} % Korrelation
\newcommand{\Normal}{\mathcal{N}} % Gaußsche Normalverteilung

\newcommand{\TODO}[1]{\textcolor{orange}{TODO: #1}}

\definecolor{StepOneColor}{rgb}{0.7,0.2,0.0}
\definecolor{StepTwoColor}{rgb}{0.2,0.7,0.0}
\definecolor{TypeInfoColor}{rgb}{0.6,0.6,0.6}

\newcommand{\stepOne}[1]{\textcolor{StepOneColor}{#1}}
\newcommand{\stepTwo}[1]{\textcolor{StepTwoColor}{#1}}
\newcommand{\typeInfo}[1]{\textcolor{TypeInfoColor}{#1}}

\title{\TODO{Der Random-Walk Metropolis-Hastings-Algorithmus für Threshold-VAR-Modelle}}
\author{Tim Baumann}
\date{29. April 2016}

\begin{document}

\begin{frame}
  \titlepage
\end{frame}

\begin{frame}
  \tableofcontents
  % TODO: mehr wie im Buch
\end{frame}

% 3.4. "The random walk MH algorithm used in a Threshold VAR model"

\section[Threshold-VAR-Modell]{Der Random-Walk Metropolis-Hastings-Algorithmus für Threshold-VAR-Modelle}

\begin{frame}[t]
  \frametitle{Das Threshold-VAR-Modell}

  % Kovarianz von $v_t$ schreiben (anstand Varianz)
  \begin{align*}
    \text{(TVAR)} \enspace
    & \begin{cases}
      Y_t = \stepOne{c_1} + \sum_{j=1}^P \stepOne{\beta_1} Y_{t-j} + v_t, \enspace
      \var(v_t) = \stepOne{\Omega_1}
      & \text{wenn } S_t \leq \stepTwo{Y^*} \\[4pt]
      Y_t = \stepOne{c_2} + \sum_{j=1}^P \stepOne{\beta_2} Y_{t-j} + v_t, \enspace
      \var(v_t) = \stepOne{\Omega_2}
      & \text{wenn } S_t > \stepTwo{Y^*}
    \end{cases} \\
    & \text{wobei } S_t \coloneqq Y_{j, t-d} \enspace \text{(Threshold-Variable)} \\
    & \typeInfo{
      Y_t, v_t, c_1, c_2 \in \R^T, \enspace
      \beta_1, \beta_2 \in \R^{T \times T}, \enspace
      \Omega_1, \Omega_2 \in \R^{T \times T}
    }
  \end{align*}

  Dabei wird die Threshold-Komponente~$j$ von~$Y$ und die Verzögerung~$d$ vom Anwender gewählt.

  \begin{bsp}<2->
    Makroökonomische Modellierung, wobei vermutet wird, dass die Stärke wirtschaftlicher Zusammenhänge (z.\,B. Multiplikator für Staatsausgaben) in Wirtschaftkrisen unterschiedlich groß ist wie in wirtschaftlich normalen oder guten Zeiten.
  \end{bsp}
\end{frame}

\begin{frame}
  \begin{prior}
    $p(Y^*) \sim \Normal(\overline{Y}^*, \sigma_{Y^*})$
    \TODO{Prior für die VAR-Parameter}
  \end{prior}

  \begin{inference}
    \begin{enumerate}
      \item Wähle Startwerte für die \stepOne{VAR-Parameter} und den Treshold $\stepTwo{Y^*}$ \\ (z.\.B. den Durschnitt oder den Median der Werte $S_t$).
      \item Gibbs-Sampling: Wiederhole die Schritte
      \begin{enumerate}
        \item
        \begin{itemize}
          \item Teile die Daten in die zwei Regime $S_t \leq Y^*$ und $S_t > Y^*$ auf.
        \end{itemize}
      \end{enumerate}
    \end{enumerate}
    Ist $Y^*$ ebenfalls bekannt, so zerfällt das Modell in zwei einfache VAR-Modelle, eines für das Regime $S_t \leq Y^*$, eines für $S_t > Y^*$.
  \end{inference}
\end{frame}


\end{document}
