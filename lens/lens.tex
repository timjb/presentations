\documentclass{beamer}

\mode<presentation> {

\usepackage{graphicx}
\usepackage[utf8]{inputenc}
\usepackage[ngerman]{babel}
\usepackage[outputdir=output]{minted} % Syntax-Highlighted Code; requires pygments to be installed

\usetheme{Boadilla}
\usecolortheme{seagull}

\setbeamertemplate{footline}[page number] % Einfacher Folienzähler als Fußzeile
\setbeamertemplate{navigation symbols}{} % Keine Navigationslinks
}

\defbeamertemplate*{title page}{customized}[1][]
{
  \centering
  \usebeamerfont{title}{\LARGE \inserttitle}\par
  \usebeamerfont{subtitle}\usebeamercolor[fg]{subtitle}\insertsubtitle\par
  \bigskip
  \usebeamercolor[fg]{titlegraphic}\inserttitlegraphic\par
  \bigskip
  \usebeamerfont{author}\insertauthor\\[0.5em]
  \usebeamerfont{institute}\insertinstitute\par
  \usebeamerfont{date}\insertdate\par
}

\title[Lens]{Lenses und Zauberwürfel}
\titlegraphic{\includegraphics[width=0.65\linewidth]{rubiks-sequence.png}}
\author{Tim Baumann}
\institute[CCA]{Curry Club Augsburg}
\date{13. August 2015}

\begin{document}

\begin{frame}
  \titlepage
\end{frame}

\iffalse
\begin{frame}
  \frametitle{Übersicht}
  \tableofcontents
\end{frame}
\fi

\begin{frame}[fragile]
  \frametitle{Was sind Lenses?}
  %\lstinputlisting[language=Haskell]{records.hs}
\begin{minted}{haskell}
data Person = Person { firstName :: String
                     , lastName :: String
                     }
data Book = Book { id :: Integer
                 , title :: String
                 , author :: Person
                 }
\end{minted}
\end{frame}

\begin{frame}
  \frametitle{Welche Bibliothek?}
  % Bildquelle: https://ro-che.info/ccc/23
  \begin{figure}
    \includegraphics[width=0.9\linewidth]{ccc-picking-lens-library.png}
    \caption{Picking a Lens Library (Cartesian Closed Comic)}
  \end{figure}
\end{frame}

\begin{frame}
  \frametitle{Multiple Columns}
  \begin{columns}[c] % The "c" option specifies centered vertical alignment while the "t" option is used for top vertical alignment
    \column{.45\textwidth} % Left column and width
    \textbf{Heading}
    \begin{enumerate}
      \item Statement
      \item Explanation
      \item Example
    \end{enumerate}

    \column{.5\textwidth} % Right column and width
    Lorem ipsum dolor sit amet, consectetur adipiscing elit. Integer lectus nisl, ultricies in feugiat rutrum, porttitor sit amet augue. Aliquam ut tortor mauris. Sed volutpat ante purus, quis accumsan dolor.
  \end{columns}
\end{frame}

\end{document}
