\documentclass{beamer}
%
% Choose how your presentation looks.
%
% For more themes, color themes and font themes, see:
% http://deic.uab.es/~iblanes/beamer_gallery/index_by_theme.html
%
\mode<presentation>
{
  \usetheme{default}      % or try Darmstadt, Madrid, Warsaw, ...
  \usecolortheme{default} % or try albatross, beaver, crane, ...
  \usefonttheme{default}  % or try serif, structurebold, ...
  \setbeamertemplate{navigation symbols}{}
  \setbeamertemplate{caption}[numbered]
} 

\usepackage[english,ngerman]{babel}
\usepackage[utf8]{inputenc}

\title{Der Endlichkeitssatz von Serre über \\ die Homotopiegruppen der Sphären}
\author{Tim Baumann}
\institute{Universität Augsburg}
\date{4. Februar 2016}

\usepackage{mathtools} % \coloneqq
\usepackage{nicefrac}
\usepackage{amsthm}
\usepackage{enumitem} % bessere Aufzählungen

\newcommand{\Z}{\mathbb{Z}} % Ganze Zahlen
\newcommand{\SC}{\mathcal{C}} % Serre-Klasse
\newcommand{\FG}{\mathcal{FG}} % Serre-Klasse der endlich erzeugten Gruppen
\newcommand{\T}{\mathcal{T}} % Serre-Klasse der endlich erzeugten Gruppen, deren Ordnung nur durch Primzahlen in einer bestimmten Menge teilbar ist
\newcommand{\F}{\mathcal{F}} % Serre-Klasse der endlichen Gruppen
\DeclareMathOperator{\Tor}{Tor} % Tor-Funktor
\newcommand{\Primes}{\mathbb{P}} % Menge der Primzahlen
\DeclareMathOperator{\coker}{coker} % Kokern

\theoremstyle{definition}
\newtheorem*{defn}{Def} % Definition
\newtheorem*{ziel}{Ziel}
\newtheorem*{satz}{Satz}
\newtheorem*{methode}{Methode}
\newtheorem*{kor}{Kor} % Korollar
\newtheorem*{bsp}{Bsp} % Beispiel
\newtheorem*{axiom}{Axiom}
\newtheorem*{lembspe}{Lem/Bspe} % Lemma/Beispiele

% Färbe \emph{} violett
\usepackage{color,graphicx,overpic}
\definecolor{Emph}{rgb}{0.2,0.2,0.8}
\renewcommand{\emph}[1]{\textcolor{Emph}{#1}}
\newcommand{\Emph}[1]{\textcolor{Emph}{#1}}
\newcommand{\TODO}[1]{\textcolor{orange}{TODO: #1}}

% http://tex.stackexchange.com/a/33177
\makeatletter
\setbeamertemplate{theorem begin}
{%
  \vspace{0.5em}
  \inserttheoremheadfont% \bfseries
  \inserttheoremname \inserttheoremnumber
  \ifx\inserttheoremaddition\@empty\else\ (\inserttheoremaddition)\fi%
  \inserttheorempunctuation ~
  \normalfont
}
\setbeamertemplate{theorem end}{%
  % empty
}
\makeatother

\begin{document}

\selectlanguage{ngerman}

\begin{frame}
  \titlepage
\end{frame}

\begin{frame}[t]
  Sei $X$ ein wegzusammenhängender topologischer Raum.
  
  \begin{onlyenv}<1->
  \begin{defn}
    Die \emph{$i$-te Homotopiegruppe} von $X$ ist
    \[
      \pi_i(X) \coloneqq [S^i, X] = \nicefrac{\{ \text{ stetige Abbildungen $S^i \to X$ } \}}{\text{Homotopie}}.
    \]
  \end{defn}
  \end{onlyenv}
  
  \begin{onlyenv}<2->
  \vspace{-1em}
  \begin{ziel}
    $\pi_i(S^n)$ studieren
  \end{ziel}
  \end{onlyenv}
  
  \begin{onlyenv}<3->
  \begin{methode}
    Verwende den \emph{Hurewicz-Homomorphismus}
    \[
      h_i : \pi_i(X) \to H_i(X), \quad
      [f : S^i \to X] \mapsto f_*(\alpha),
    \]
    wobei $\alpha \in H_i(S^i)$ ein fester Erzeuger ist, und
  \end{methode}
  
  \begin{satz}[Hurewicz-Thm]
    Sei $n \geq 2$.

    \begin{tabular}{l l}
      Angenommen, & $\pi_i(X) = 0$ \enspace für $i < n$. \\
      Dann gilt & $\widetilde{H}_i(X) = 0$ \enspace für $i < n$ \\
      und & $h_n : \pi_n(X) \to H_n(X)$ ist ein Isomorphismus.
    \end{tabular}
  \end{satz}
  \end{onlyenv}
  
  \begin{onlyenv}<4->
    \begin{kor}
      $\pi_i(S^n) = 0$ für $i < n$, \quad
      $\pi_n(S^n) \cong \Z$ für $n \geq 2$.
    \end{kor}
  \end{onlyenv}
\end{frame}

\begin{frame}
  \begin{satz}[Jean-Pierre Serre, 1951] \mbox{} \\
    Die Gruppen $\pi_i(S^n)$, $i > n$, sind endlich bis auf die Gruppen $\pi_{2n-1}(S^n)$ für $n \geq 2$ gerade, welche isomorph zur direkten Summe von $\Z$ und einer endlichen Gruppe sind.
  \end{satz}
  
  \begin{visibleenv}<2->
  \begin{bsp}
    Die Hopf-Faserung $\eta : S^3 \to S^2$ ist ein Element der Ordnung unendlich in $\pi_3(S^2)$.
  \end{bsp}
  \end{visibleenv}
\end{frame}

\begin{frame}
  \frametitle{Erste Verallgemeinerung: Serre-Klassen}
  \begin{defn}
    Eine Klasse $\SC$ von abelschen Gruppen heißt \emph{Serre-Klasse}, falls
    \begin{enumerate}[label=(\Roman*)]
      \item Für jede exakte Seq. $A \to B \to C$ von abelschen Gruppen gilt:
      \[
        B \in \SC \iff A, C \in \SC.
      \]
      \item Für $A, B \in \SC$ sind auch $A \otimes B \in \SC$ und $\Tor(A, B) \in \SC$.
    \end{enumerate}
  \end{defn}
  \begin{axiom} \mbox{} \\
    \begin{enumerate}[label=(\Roman*)]
      \setcounter{enumi}{2}
      \item Es sei $G \in \SC$.
      Dann ist $K(G, n)$ $\SC$-azyklisch für alle $n \geq 1$.
    \end{enumerate}
  \end{axiom}

  \begin{visibleenv}<2->
  \begin{lembspe}
    Folgendes sind Serre-Klassen, die Axiom (III) erfüllen:
    \begin{enumerate}[label=\alph*)]
      \item $\T_P \coloneqq \left\{ \begin{array}{l}
        \text{endl. ab. Gruppen, deren Ordnung ein Produkt} \\
        \text{von Primzahlen in $P \subseteq \Primes$ ist}
      \end{array} \right\}$,
      %wobei $\Primes$ die Menge aller Primzahlen bezeichnet,
      \item $\F \coloneqq \T_\Primes = \{\, \text{endliche abelsche Gruppen} \,\}$
      \item $\FG \coloneqq \{\, \text{endlich erzeugte abelsche Gruppen} \,\}$
    \end{enumerate}
  \end{lembspe}
  \end{visibleenv}
\end{frame}

\begin{frame}
  \frametitle{Erste Verallgemeinerung: Serre-Klassen}
  \begin{satz}[Hurewicz-mod-$\SC$-Thm] Es sei $n \geq 2$. \\
    Es sei $\SC$ eine Serre-Klasse, die (III) erfüllt. \\
    Es sei $X$ ein einfach zusammenhängender topologischer Raum. \\[0.5em]
    
    \begin{tabular}{l l}
      Angenommen, & $\pi_i(X) \in \SC$ \enspace für $i < n$. \\
      Dann gilt & $\widetilde{H}_i(X) \in \SC$ \enspace für $i < n$ \\
      und & $h_n : \pi_n(X) \to H_n(X)$ ist ein Isomor. modulo $\SC$, \\
      & d.\,h. $\ker(h_n) \in \SC$ und $\coker(h_n) \in \SC$.
    \end{tabular}
  \end{satz}
  
  \begin{visibleenv}<2->
    \begin{kor}
      Sei $n \geq 2$.
      Dann sind die Homotopiegruppen $\pi_i(S^n)$, $i \geq 1$ endlich erzeugt.
    \end{kor}
  \end{visibleenv}
\end{frame}

\begin{frame}
  \frametitle{Zweite Verallgemeinerung: Relativität}

  \begin{satz}[relatives Hurewicz-mod-$\SC$-Thm] Es sei $n \geq 2$. \\
    Es sei $\SC$ eine Serre-Klasse, die (III) erfüllt. \\
    Es sei $(X, A)$ ein einfach zusammenhängendes Raumpaar mit $A \neq \emptyset$ und $A$ einfach zusammenhängend. \\[0.5em]
    
    \begin{tabular}{l l}
      Angenommen, & $\pi_i(X, A) \in \SC$ \enspace für $i < n$. \\
      Dann gilt & $H_i(X, A) \in \SC$ \enspace für $i < n$ \\
      und & $h_n : \pi_n(X, A) \to H_n(X, A)$ ist ein Isomor. modulo $\SC$.
    \end{tabular}
  \end{satz}
  
  \begin{visibleenv}<2->
    \begin{kor}
      Es sei $f : A \to B$ stetig, $A$ und $B$ nichtleer und einfach zusammenhängend.
      Dann sind äquivalent:
      \begin{enumerate}
        \item[a)] $f_* : \pi_i(A) \to \pi_i(B)$ \enspace
        \begin{minipage}[t]{0.65 \linewidth}
          ist ein Isomorphismus mod~$\SC$ für $i < n$ \\
          und ein Epimorphismus mod~$\SC$ für $i = n$.
        \end{minipage}
        \item[b)] $f_* : H_i(A) \to H_i(B)$ \enspace
        \begin{minipage}[t]{0.65 \linewidth}
          ist ein Isomorphismus mod~$\SC$ für $i < n$ \\
          und ein Epimorphismus mod~$\SC$ für $i = n$.
        \end{minipage}
      \end{enumerate}
    \end{kor}
  \end{visibleenv}
\end{frame}

\end{document}
